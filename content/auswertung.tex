\section{Auswertung}
\label{sec:Auswertung}

\begin{table}[!htp]
\centering
\caption{Die Amplituden einer gedämpften harmonischen Schwingung zu den jeweiligen Zeitpunkten zur Bestimmung des Dämpfungswiderstandes.}
\label{tab:zeit-amplitude}
\begin{tabular}{S[table-format=3.0] S[table-format=1.2] S[table-format=2.2]}
\toprule
{$t$ / µs} & {$U_\text{C}$ / V} & {$\ln\bigg(\frac{U_\text{C}}{1 \si{V}}\bigg)$} \\
\midrule
14 &  8.16 & 2.10 \\
27 &  6.80 & 1.92 \\
41 &  5.92 & 1.78 \\
54 &  4.96 & 1.60 \\
68 &  4.40 & 1.48 \\
81 &  3.60 & 1.28 \\
95 &  3.20 & 1.16 \\
108 & 2.72 & 1.00 \\
122 & 2.32 & 0.84 \\
136 & 2.00 & 0.69 \\
149 & 1.68 & 0.52 \\
162 & 1.44 & 0.36 \\
176 & 1.28 & 0.25 \\
189 & 1.12 & 0.11 \\
203 & 0.96 & -0.04 \\
217 & 0.80 & -0.22 \\
\bottomrule
\end{tabular}
\end{table}

\begin{figure}
    \centering
    \includegraphics{build/plot-amplitude.pdf}
    \label{fig:zeit-amplitude}
    \caption{Plot und Fit der Messwerte der Messung der Amplitude in Zeitabhängigkeit.}
\end{figure}

\begin{center}
    $a = (1.131 \pm 0.008) \cdot 10^{4} \frac{\ln(\symup{V})}{\symup{s}}$

    $b = (2.23 \pm 0.01) \ln(\symup{V})$
\end{center}

%%%%%%%%%%%%%%%%%%%%%%%%%%%%%%%%%%%%%%%%%%%%%%%%%%%%%%

\begin{table}[!htp]
\centering
\caption{Die angegebenen Gerätedaten.}
\label{tab:geraetedaten}
\begin{tabular}{S[table-format=1.2] @{${}\pm{}$} S[table-format=1.2] S[table-format=1.3] @{${}\pm{}$} S[table-format=1.3] S[table-format=2.1] @{${}\pm{}$} S[table-format=1.1] S[table-format=3.1] @{${}\pm{}$} S[table-format=1.1]}
\toprule
\multicolumn{2}{c}{$L$ / mH} & \multicolumn{2}{c}{$C$ / nF} & \multicolumn{2}{c}{$R_1$ / $\symup{\Omega}$} & \multicolumn{2}{c}{$R_2$ / $\symup{\Omega}$} \\
\midrule
3.53 & 0.03 & 5.015 & 0.015 & 30.3 & 0.1 & 271.6 & 0.3 \\
\bottomrule
\end{tabular}
\end{table}

Errechneter $R_\text{ap}$ Wert

\begin{center}
    $R_\text{Theorie} = (1677 \pm 8)$ $\symup{\Omega}$ 
\end{center}

Gemessener Wert

\begin{center}
    $R_\text{ap} = (1355 \pm 5)$ $\symup{\Omega}$
\end{center}

%%%%%%%%%%%%%%%%%%%%%%%%%%%%%%%%%%%%%%%%%%%%%%%%%%%%%%

\begin{table}[!htp]
\centering
\caption{Die Amplituden und Phasenverschiebung in Frequenzabhänigkeit mit zuvor gemessenem $U_0 = 6,\~55$ V.}
\label{tab:var-freq}
\begin{tabular}{S[table-format=2] S[table-format=2.1] S[table-format=1.2] S[table-format=1.2]}
\toprule
{$f$ / kHz} & {$U_\text{C}$ / V} & {$\frac{U_\text{C}}{U_0}$} & {$\Delta t$ / µs} \\
\midrule
12 &  7.2 & 1.10 & 1.2 \\
14 &  7.8 & 1.19 & 1.5 \\
16 &  8.1 & 1.24 & 1.5 \\
18 &  8.3 & 1.27 & 1.6 \\
20 &  9.0 & 1.37 & 1.7 \\
22 &  9.5 & 1.45 & 1.8 \\
24 & 10.3 & 1.57 & 2.2 \\
26 & 11.3 & 1.73 & 2.5 \\
28 & 12.4 & 1.89 & 2.9 \\
30 & 13.8 & 2.11 & 3.4 \\
32 & 15.2 & 2.32 & 4.2 \\
33 & 16.0 & 2.44 & 4.6 \\
34 & 16.5 & 2.52 & 5.1 \\
35 & 16.8 & 2.56 & 5.6 \\
36 & 16.8 & 2.56 & 6.3 \\
37 & 16.5 & 2.52 & 6.8 \\
38 & 15.9 & 2.43 & 7.3 \\
39 & 15.1 & 2.31 & 7.7 \\
40 & 14.2 & 2.17 & 8.0 \\
41 & 13.2 & 2.02 & 8.2 \\
42 & 12.2 & 1.86 & 8.3 \\
44 & 10.5 & 1.60 & 8.5 \\
46 &  9.0 & 1.37 & 8.5 \\
48 &  7.5 & 1.15 & 8.4 \\
50 &  6.5 & 0.99 & 8.28 \\
52 &  5.7 & 0.87 & 8.08 \\
54 &  5.1 & 0.78 & 7.84 \\
56 &  4.6 & 0.70 & 7.72 \\
58 &  4.1 & 0.63 & 7.48 \\
60 &  3.7 & 0.56 & 7.28 \\
62 &  3.4 & 0.52 & 7.08 \\
\bottomrule
\end{tabular}
\end{table}

\begin{figure}
    \centering
    \includegraphics{build/plot-guete.pdf}
    \label{fig:guete}
    \caption{Plot und Theoriekurve des Amplitudenverhältnis gegen die Frequenz aufgetragen.}
\end{figure}

\begin{center}
    $q_\text{exp} = 2.56$
\end{center}

\begin{center}
    $q_\text{theo} = 3.09 \pm 0.01$
\end{center}

\begin{equation}
    \Delta q_\text{theo} = \sqrt{\cdot \frac{L}{C}} \cdot \sqrt{ \bigg( \frac{1}{R^2} \Delta R \bigg)^2 + \bigg(\frac{1}{2RC} \Delta L \bigg)^2 + \bigg(\frac{L}{2RC^2} \Delta C \bigg)^2}
\end{equation}

\begin{center}
    $\omega_+ = (2.79 \pm 0.01) \cdot 10**{5}$ $\frac{1}{s}$

    $\omega_- = (2.022 \pm 0.008) \cdot 10**{5}$ $\frac{1}{s}$
\end{center}

Der Fehler nach Gauß errechnet sich dabei nach

\begin{equation}
    \Delta \omega_{+,-} = \sqrt{\bigg( \frac{\partial \omega_{+,-}}{\partial C} \cdot \Delta C \bigg)^2 + \bigg( \frac{\partial \omega_{+,-}}{\partial L} \cdot \Delta L \bigg)^2 + \bigg( \frac{\partial \omega_{+,-}}{\partial R} \cdot \Delta R \bigg)^2}, 
\end{equation}

wobei die einzelnen partiellen Ableitungen im Folgenden aufgeführt sind:

\begin{center}
    $\bigg( \frac{\partial \omega_{+,-}}{\partial C} \bigg)^2 = \Bigg( \dfrac{1}{2L\sqrt{\frac{1}{LC}+\frac{R^2}{4L^2}}C^2} \Bigg)^2$

    $\bigg( \frac{\partial \omega_{+,-}}{\partial R} \bigg)^2 = \Bigg( \dfrac{R}{4L^2\sqrt{\frac{R^2}{4L^2}+\frac{1}{CL}}} \pm \dfrac{1}{2L} \Bigg)^2$

    $\bigg( \frac{\partial \omega_{+,-}}{\partial L} \bigg)^2 = \Bigg( \mp \dfrac{R}{2L^2}+\dfrac{-\frac{1}{CL^2}-\frac{R^2}{2L^3}}{2\sqrt{\frac{1}{CL}+\frac{R^2}{4L^2}}} \Bigg)^2$.
\end{center}

\begin{figure}
    \centering
    \includegraphics{build/plot-phase.pdf}
    \label{fig:phase}
    \caption{Plot und Theoriekurve des Phasenunterschiedes gegen die Frequenz aufgetragen.}
\end{figure}
